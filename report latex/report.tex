\documentclass[12pt, landscape]{article}
\usepackage[a2paper]{geometry}
\usepackage[numbers]{natbib}
\usepackage{multicol}
\usepackage{blindtext}
\usepackage{setspace}
\setlength{\columnsep}{1cm}

%\doublespacing
% \documentclass[12pt,landscape]{article}
% \usepackage[a3paper]{geometry}

\title{Road Sign Recognition with CNN and RCNN}
\author{Aksel Tahir 6548051}
\date{11.05.2021}

\begin{document}
\maketitle
\tableofcontents
\pagebreak

\begin{multicols}{3}
\section{Introduction}
Road signs are a present system in virtually all road infrastructure. They are
of critical importance to interpreting correct road usage, road regulations and
route recommendations. Their presence is integral to the safe and functional
road use.

Contemporary road signs follow strict design rules to optimise their clarity of
intention. These rules allow them to be as easy as possible for human
interpretation. However, humans are prone to distraction, misinterpretation and
other general mistakes, which is why road sign recognition (RSR) algorithms
are a fast-advancing point of development in autonomous driving research.

Standard computer vision methods are not versatile enough to deal with the
plethora of different physical road conditions. This is why applying a deep
learning approach to the problem is necessary - A well crafted AI can exceed
even human vision in RSR.

In this project we propose an RSR solution using several different neural
network models and evaluating their performance. Since standard computer vision
methods are not versatile enough to deal with the plethora of different physical
road conditions, it is necessary to apply a deep learning approach to the
problem - A well crafted AI can exceed even human vision in RSR.

\section{Related Work}
With the advent of AI computing autonomous and assisted driving has been an area
of extensive research. Road sign recognition (RSR) systems are integral to the
field. The functional implementation of RSR systems depends on two related
issues - Road sign detection (RSD) and road sign classification (RSC). RSD
pertains to localising the relevant information from the data, and RSC to
identifying the data with its correct labels. Lots of outstanding results for
the detection and classification of traffic signs have been proposed in
\citep{Classification1}, \citep{Classification2}, \citep{Classification3},
\citep{Classification4}, \citep{Classification5}, 

\section{Architectures}
\subsection{CNN}
\subsection{R-CNN}
\subsection{Faster R-CNN}
    
\section{Dataset}
\subsection{Loading the dataset}
\subsection{Preprocessing data}

\section{Implementation}
\subsection{CNN}
\subsection{R-CNN}
\subsection{Faster R-CNN}

\section{Analysis and Evaluation}

\section{Conclusions}

% \section{References}

\bibliographystyle{unsrt}
\bibliography{References}
\nocite{*}
\end{multicols}
\end{document}