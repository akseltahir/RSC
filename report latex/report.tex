\documentclass[12pt]{article}
\usepackage{geometry}
\usepackage{natbib}

% \documentclass[12pt,landscape]{article}
% \usepackage[a3paper]{geometry}

\title{Road Sign Classification and Detection with CNN and RCNN [Recognition?]}
\author{Aksel Tahir 6548051}
\date{11.05.2021}


\begin{document}
\maketitle

\tableofcontents

\section{Introduction}

A fundamental part of all road infrastructure is the traffic sign system. Any
road intended for vehicular use is incomplete without the implementation of a
traffic sign system. They are of critical importance to interpreting correct
road usage, road regulations and route recommendations. Their presence is
integral to the safe and functional road use.

Contemporary road signs follow strict design rules to optimise their clarity of
intention. These rules allow them to be as easy as possible for human
interpretation. However, humans are prone to distraction, misinterpretation and
other general mistakes, which is why developing road sign classification
(RSC[?]) algorithms is a huge part of the automation of driving. Standard
computer vision methods are not versatile enough to deal with the plethora of
different physical conditions on roads. This is why applying a deep learning
approach to the problem is necessary - A well crafted AI can exceed even human
vision in RSC[?].

In this project we propose an RSC[?] system by applying several neural network
models and evaluating their performance. [talk about CNN, RCNN, YOLO, other
models we'll use].


\section{Related Work}
With the advent of AI computing autonomous and assisted driving has been an area
of extensive research. Road sign recognition (RSR) systems are integral to the
field. The functional implementation of RSR systems depends on two related
issues - Road sign detection (RSD) and road sign classification (RSC). RSD
pertains to localising the relevant information from the data, and RSC to
identifying the data with its correct labels.
\cite[1]{rachmadi} 

\section{Architectures}
\subsection{CNN}
\subsection{GNN}
\subsection{RCNN}
\subsection{Faster-RCNN}
    
\section{Dataset}
\subsection{Loading the dataset}
\subsection{Preprocessing data}

\section{Implementation}
\subsection{CNN}
\subsection{GNN}
\subsection{Faster-RCNN}

\section{Analysis and Evaluation}

\section{Conclusions}

% \section{References}

\bibliographystyle{IEEEtranN}
\bibliography{References}

\end{document}